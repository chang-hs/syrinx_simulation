\documentclass[fleqn,10pt]{wlscirep}
\usepackage[utf8]{inputenc}
\usepackage[T1]{fontenc}
\usepackage{newfloat}
\DeclareFloatingEnvironment{video}
\title{Direction-Selective Resistance to Cerebrospinal-Fluid Flow As
the Mechanism of Syrinx Generation in Syringomyelia}

\author{Han Soo Chang, M.D.}
\affil{Department of Neurosurgery, Tokai University\\Isehara, Japan}

\keywords{syringomyelia, pathophysiology, simulation}
\input{embed_video.tex}
\begin{abstract}
    The pathophysiology of syringomyelia is not well understood. The main
    theoretical problem is how cerebrospinal fluid (CSF) enters from
    low-pressure subarachnoid space to high-pressure syrinx and remains
    inside the syrinx. We approached this problem with computer simulation
    of the CSF flow in the spine.
\end{abstract}
\begin{document}

\flushbottom
\maketitle
% * <john.hammersley@gmail.com> 2015-02-09T12:07:31.197Z:
%
%  Click the title above to edit the author information and abstract
%
\thispagestyle{empty}


\section*{Introduction}

The pathophysiology of syringomyelia is still poorly understood. A number
of hypotheses exist in the literature \cite{gardner1958mechanism,
williams1980pathogenesis, milhorat1999chiari, ball1972pathogenesis,
klekamp2002pathophysiology, duboulay1974mechanism, heiss1999elucidating,
milhorat1993anatomical, stoodley2000pathophysiology, terae1994increased,
chang2003hypothesis, chang2004theoretical, greitz2006unraveling}, but they
provide widely different explanations on the mechanisms of syrinx
generation.  Nevertheless, most researchers seem to agree on the following
points. First, the syrinx fluid is identical to the CSF, and there should
be some communication between the syrinx and the subarachnoid space. This
point is supported by many studies \cite{ellertsson1969syringomyelia,
ellertsson1969myelocystographic, li1987conventional, heiss2019origin},
although a few different opinions exist \cite{greitz2006unraveling,
koyanagi1997surgical}. Second, some derangement of CSF flow in the spinal
subarachnoid space causes syrinx both in Chiari-I malformation
\cite{wolpert1994chiari, bhadelia1995cerebrospinal, heiss1999elucidating,
hofmann2000phasecontrast, quigley2004cerebrospinal} and subarachnoid
arachnopathy \cite{klekamp1997treatment, brodbelt2003altered,
heiss2012pathophysiology, chang2014dorsal}. Notably, the cerebellar tonsil
deranges the CSF flow in the former and adhesive arachnoiditis in the
latter.

The problem, however, is where this communicating channel resides and what
mechanism generates the syrinx. On these points, there is no solid
experimental or clinical evidence, and the opinions of researchers deviate
widely. Gardner et al. \cite{gardner1958mechanism} thought that the central
canal intercommunicates the syrinx and the fourth ventricle, and arterial
pressure waves exerted on the central canal generate the syrinx. Williams
et al. also postulated the communication through the central canal, but for
the syrinx generation mechanism, he emphasized the craniospinal pressure
gradient produced by Valsalva maneuver et al
\cite{williams1980pathogenesis}.

On the other hand, Ball and Dayan \cite{ball1972pathogenesis} assumed that
CSF enters the syrinx through the perivascular space of arteries
penetrating the spinal cord. This idea has the following variations. Heiss
et al.  \cite{heiss1999elucidating} proposed that the piston-like movement
of the cerebellar tonsils in Chiari-I patients generates pressure waves in
the spinal subarachnoid space, which subsequently drive CSF into the syrinx
through the perivascular space. Stoodley et al. also considered the
perivascular space as the communicating channel, but he assumed the
arterial pulse pressure as the driving force of CSF
\cite{stoodley2000mechanisms} All these assumptions are not proven and
remain hypothetical. Although the perivascular-space theory seems to be
favored by recent researchers, there remains the possibility that a thin
communicating channel exists between the syrinx and the fourth ventricle
\cite{chang2021hypothesis}. 

In our opinion, the main theoretical problems reside in the following
points.
\begin{enumerate}
    \item No theory can explain the pathophysiological mechanism of
syringomyelia in a unified fashion.
    \item No theory can explain how CSF enters from the low-pressure subarachnoid space to the high-pressure syrinx cavity and remains inside.
\end{enumerate}

As to the first point, there are different syringomyelia types, such as
Chiari-I-malformation and spinal-arachnopathy-related
\cite{klekamp1997treatment}. The Chiari-I-malformation-related
syringomyelia is further divided into communicating and non-communicating
\cite{elliott2013syringomyelia}. For each of them, current theories assume
a distinct mechanism of syrinx generation. However, it may be more natural
to conjecture some common mechanism underlying these different types of
syringomyelia \cite{stoodley2000mechanisms}.  The second point is
theoretically essential but challenging to solve. Physical theories dictate
that the expanded syrinx cavity has higher pressure than the subarachnoid
space \cite{serwayr.a.2016fluids, heiss1999elucidating,
davis1989mechanisms, ellertsson1970distending}. Therefore, merely assuming
a communicating channel does not explain how CSF enters the syrinx and
remains inside against this pressure gradient. Even if we take a specific
time window where the subarachnoid pressure exceeds the syrinx pressure, it
does not explain how the CSF remains inside the syrinx after it.

The current article is part of our effort to solve the above theoretical
problems. In our previous paper \cite{chang2021hypothesis}, we hypothesized
that if there is a resistance to CSF flow in a particular direction
(rostral or caudal), it causes a one-way valve-like effect on a CSF channel
inside the spinal cord, leading to the accumulation of CSF and generation
of a syrinx. This hypothesis was attractive to us because it could explain
the pathophysiology of both Chiari-I-malformation-related syringomyelia and
arachnopathy-related syringomyelia. Namely, in
Chiari-I-malformation-related syringomyelia, the herniated cerebellar
tonsils may function as a direction-selective resistance. In
arachnopathy-related syringomyelia, some arachnoid adhesion may play the
same function \cite{chang2014dorsal}.  However, in that article, we just
drew a rough sketch of this process and left out a detailed explanation.
The current article will describe in detail how direction-selective
resistance in the subarachnoid space generates a one-way valve mechanism in
the CSF channel inside the spinal cord.

For this purpose, we used a mathematical model simulating the CSF movement
of the spine---a revised version of our previous model \cite{chang2003hypothesis,
chang2004theoretical}. This model describes the spinal CSF movement as an
electric current in a modeled circuit (a lumped parameter model with
multiple compartments \cite{shi2011review}), and it assumes the existence of a
patent central canal. We placed a direction-selective resistance in this
model at a certain point in the spinal subarachnoid space and observed how
it affects the CSF flow in the central canal.

\section*{Material and Method}

Some problem exists in the computer simulation of the motion of biological
fluids such as blood and CSF. Such bodily fluids move inside flexible
tubes. It, therefore, requires different analysis techniques from those
used in the engineering field, where the boundary of the conduit is
supposed to be solid. For this purpose, researchers widely used lumped
parameter models \cite{shi2011review, kokalari2013review}. This model
considers the fluid flow inside a flexible tube in analogy to the flow of
electricity in an electric circuit. The accumulation of electricity in a
capacitor represents the expansion of a flexible tube and the accompanying
pressure elevation. An electrical resistor represents the frictional
resistance to flow. This model has a wide variety. For example, it may
model the whole cardiovascular system as one electric circuit, or it may
model it as a synthesis of multiple compartments of electric circuits
\cite{shi2011review, kokalari2013review}. Our current model is the type of a
lumped parameter model with multiple compartments.  In this study, we did
not intend to make a quantitatively precise model of the spinal CSF flow
but to make a basic model that reveals the phenomenon underlying the
generation of syringomyelia. For this purpose, we adopted a revised version
of our previous lumped parameter model with multiple compartments.

Previously, we developed a mathematical model that simulated the CSF flow
in the spine \cite{chang2003hypothesis, chang2004theoretical}. This model (a
multiple-compartment version of a 1-dimensional lumped parameter model)
could describe the CSF movement in the whole spine (Figure \ref{fig:model}).

\begin{figure}[ht]
    \centering
    \includegraphics[width=\textwidth]{ps_circuit_schema.jpg}
    \caption{Schema of the electric circuit model of the CSF dynamics in the
    spine.}
    \label{fig:model}
\end{figure}

A set of differential equations can describe the behavior of this model.
Using computer software, we can numerically calculate its behavior to a
cranial pressure wave (defined as a boundary condition on the cranial
points).  This time, we improved the previous model as follows.

\begin{itemize}
    \item We increased the number of compartments from 10 to 100,
        thereby making the model more precise.
    \item We estimated the values of the parameters
        (the capacitance and resistance of each component)
        of the model as follows so that the model will become more realistic.
        \begin{itemize}
            \item First, we set the length of the modeled spinal cord to be 1 meter.
            \item The resistance of the subarachnoid space ($R$) was
                estimated using the following equations of Poisseuille
                \cite{brook1999numerical, sherwin2003computational, huilgol2020fast}. 
             $$\Delta P=\frac{8\pi \mu{}LQ}{A^{2}}=RQ$$
                \begin{itemize}
                    \item $\Delta P$: Pressure difference between the adjacent compartments
                    \item Q: flow speed per unit surface
                    \item $\mu$: viscosity coefficient. In this case, it was set to the value of water (0.0007).
                    \item L: distance between the adjacent compartments. It was set to 1 cm.
                    \item A: cross sectional area of the subarachnoid
                        space. It was set to the value of a concentric
                        annulus \cite{huilgol2020fast} with the outer diameter of 1cm and the inner diameter of 0.7 cm ($1.6\times{}10^{-4} (m^2)$) 
                \end{itemize}
            \item Thus, $R$ was calculated to be $6872\hspace{0.2cm}(Pa\cdot{}sec/m^3)$
            \item The resistance of the central canal ($r$) was estimated
                using the same equation with $A$ set to
                $\pi{}(10^{-4})^2\hspace{0.1cm}(m^2)$, i.e. the cross
                sectional area of a tube with a diameter of 100 $\mu{}m$.
                Thus, $r$ was calculated to be $1.78\times10^{11}\hspace{0.2cm}(Pa\cdot{}sec/m^3)$
        \end{itemize}
    \item We determined the capacitance ($C_{sub}$)corresponding to the dural elasticity so that the pressure-wave velocity determined by the time constant ($RC$) will roughly correspond to the pressure-wave velocity of the downward CSF wave observed in phase-contrast MRI of normal individuals. Thus, we set $C_{sub}=0.1\hspace{0.2cm}(m^3/Pa\cdot{}sec)$.
\end{itemize}

Figure 1 shows the scheme of the constructed electric circuit model. This
model represents the CSF movement in the spine as electric flow through
multiple compartments of capacitances connected with resistors. Table 1
shows the values of the resistors and capacitors of the system.  A set of
differential equations can describe the behavior of this electric circuit,
and we can solve it numerically by setting the voltage at the cranial nodes
as the boundary condition (Figure \ref{fig:circuit}). In the previous articles
\cite{chang2003hypothesis, chang2004theoretical}, we only analyzed the
transient behavior of the model to a sudden pressure increase on the
cranial side of the subarachnoid space. This analysis helped simulate the
situation of coughing or Valsalva maneuvers. In this article, however, we
analyzed the steady-state response of the model to an oscillating cranial
pressure wave simulating the normal cardiac pulsation of the CSF.

\begin{figure}[ht]
    \centering
    \includegraphics[width=\textwidth]{electric_circuit_new.jpg}
    \caption{Electric circuit diagram representing the CSF dynamics of the spine}
    \label{fig:circuit}
\end{figure}

We numerically solved the differential equations using computer software
(Mathematica version 12, Wolfram Research, Champaign, IL, U.S.A.). We set
the boundary conditions as follows. (1) The voltage at the two cranial
nodes was set to a sine wave oscillating around 100 mmHg with an amplitude
of 20 mmHg at one cycle per second. (2) The initial dural pressure was set
at 100 mmHg in all segments. We set the step of the numerical solution to
1/5000 second and calculated the solution from zero to 20 seconds. We
displayed the obtained solution as a movie in mp4 format.

We analyzed the original normal system and two of its modifications. In the
first modification, we simply increased the subarachnoid resistance at
point 25 ($R_{25}$) by 20 times. In the second modification, we placed a
direction-selective subarachnoid resistance at point 25, so that only the
resistance to the caudal flow would be increased by 20 times.

\section*{Results}

We present the systems' responses to twenty cycles of to-and-fro waves as
animations. The x-axis of each animation represents the 100 nodes of the
circuit laid out from the cranial to caudal direction. The y-axis displays
some of the following four measures: the dural tension (voltage in D
capacitors in Figure 2), the canal tension (voltages in C capacitors), the
subarachnoid CSF flow (flows in R resistors), and the canal flow (flows in
r resistors). We plotted the caudal flow in the positive and the rostral
flow in the negative direction. The pressure values are shown in $cmH_2O$,
the flows in $ml/sec$. To plot the four measures in a single animation, we
multiplied the following three measures---the canal tension, the
subarachnoid CSF flow, and the canal flow---with the following
coefficients, respectively: namely, the canal tension with 50, the
subarachnoid flow with 0.005, and the canal flow with $1.5\times10^{5}$. 

Video \ref{video:norm} shows the original system's response representing
the normal condition. In this state, CSF makes a smooth to-and-fro movement
in the subarachnoid space with the corresponding pressure wave along the
dura and the central canal. CSF also makes to-and-fro movements in the
canal.

\begin{video}[hbt]
    \embedvideo*{\includegraphics[width=\textwidth]{thumbnail_norm.jpg}}{totalAnimationNorm.mp4}
    \caption{Video showing the normal circuit response}
    \label{video:norm}
\end{video}

In Video \ref{video:simple_block}, we increased the subarachnoid resistance
$R_{25}$ by 20 times, thereby simulating a simple block of the subarachnoid
flow in both directions. In this condition, both the caudal and rostral
flow along the resistance produced a drop of the dural tension downstream
to the resistance, corresponding to the increased resistance both in the
caudal and rostral flow phases. This pressure drop caused an increase in
canal flow in the same direction as the subarachnoid flow. This increased
canal flow caused a transient increase in the canal pressure distal to the
block and a decrease proximal to the block. However, these pressure changes
alternated along the alternation of the flow direction and did not produce
a sustained pressure increase.

\begin{video}[hbt]
    \embedvideo*{\includegraphics[width=\textwidth]{thumbnail_simple_block.jpg}}{totalAnimationSb.mp4}
    \caption{Video showing the response of the circuit with a simple
    resistor at point 25}
    \label{video:simple_block}
\end{video}

\begin{video}[hbt]
    \embedvideo*{\includegraphics[width=\textwidth]{thumbnail_oneway.jpg}}{totalAnimation.mp4}
    \caption{Video showing the response of the circuit with a
    direction-selective resistor at point 25}
    \label{video:oneway}
\end{video}

\begin{video}[hbt]
    \embedvideo*{\includegraphics[width=\textwidth]{thumbnail_oneway.jpg}}{canalFlowOneway.mp4}
    \caption{Video showing the canal flow in the model with
    direction-selective resistance at point 25}
    \label{video:canal_flow_oneway}
\end{video}

\begin{video}[hbt]
    \embedvideo*{\includegraphics[width=\textwidth]{thumbnail_oneway.jpg}}{canalFlowComparison.mp4}
    \caption{Video showing the canal flow in the two models: simple
    resistance at point 25 and direction-selective resistance at point 25}
    \label{video:canal_comparison}
\end{video}

In Video \ref{video:oneway}, we replaced the subarachnoid resistor $R_{25}$
with a direction-selective resistor whose resistance to the rostral flow
was unchanged but that to the caudal flow was increased by 20 times.  This
time, as shown in Video \ref{video:oneway}, sustained high pressure
appeared in the central canal in the segment distal to the replaced
resistor, and sustained low pressure in the segment proximal to it.  This
sustained pressure gradually accumulated as the flow cycle proceeded.  The
dural tension showed a pressure drop at node 25 only during the caudal-flow
phase.  The to-and-fro canal flow increased near the node 25 similarly to
that in the simple block above, but, this time, the increase was larger in
the caudal direction.  

We took out the canal flow and showed it in Video 4.  Observing this video,
we can see that the cumulative total of the caudal flow is larger than that
of the rostral flow, and it means that the CSF is virtually pumped caudally
at node 25. To further elucidate this point, we plotted the canal flow in
the simple block as shown in Video \ref{video:simple_block} and that in the
one-way block as shown in Video \ref{video:oneway}. We can clearly see in
this video that the caudal flow in the canal is 


\section*{Discussion}

In this article, we theoretically analyzed the CSF movement in the spinal
cord using a lumped parameter model with multiple components. It simulated
a system with an elastic tube (dura) containing an elastic cylindrical
material (spinal cord) that itself had a fluid channel inside (the central
canal). When we placed a direction-selective resistor in the subarachnoid
space and evoked a to-and-fro pressure wave on this system, it produced a
sustained pressure elevation in the segment distal to the
direction-selective resistor. This phenomenon may explain the pathogenesis
of syringomyelia both in Chiari I malformation and syringomyelia associated
with arachnopathy.

Subarachnoid pressure affects the pressure inside the central canal. As
seen in Figure \ref{fig:circuit}, the absolute pressure inside the central
canal is the sum of the subarachnoid and canal pressure (voltages of $C_k$
and $D_k$ in electrical terms). Suppose a one-way valve selectively resists
caudal flow in the subarachnoid space at point A. Caudal CSF flow creates a
pressure drop across this valve, with the distal pressure smaller than the
proximal one. It decreases the absolute canal pressure distal to point A
because it is the sum of the subarachnoid pressure and the canal pressure.
It, therefore, creates a pressure gradient in the central canal across
point A, thereby increasing the distal CSF flow in the canal at that point
(Figure \ref{fig:pump_close}).

\begin{figure}[hbt]
    \centering
    \includegraphics[width=\textwidth]{pumping_mechanism_close.jpg}
    \caption{Increased central-canal flow across a direction-selective
    resistance in the subarachnoid space}
    \label{fig:pump_close}
\end{figure}

On the contrary, reverse flow, not encountering resistance, does not create
a pressure drop (Figure \ref{fig:pump_open}). Although some of the CSF that
had been pumped caudally during the caudal-flow phase will flow back
rostrally, its amount will be smaller. The net result will be that some CSF
is pumped caudally in one cycle of the to-and-fro movement. Thus, CSF
gradually accumulates in the distal segment of the resistance (Video 3). We
hypothesize that this is the mechanism underlying the syrinx generation.

\begin{figure}[hbt]
    \centering
    \includegraphics[width=\textwidth]{pumping_mechanism_open.jpg}
    \caption{Normal central-canal flow across a direction-selective
    resistance during the reverse flow}
    \label{fig:pump_open}
\end{figure}

This hypothesis solves the theoretical problems pointed out in the
Introduction. A one-way valve in subarachnoid space creates an asymmetry of
the pressure gradient between the caudal and rostral flow phases in the
central canal. This asymmetric alternation of pressure gradient effectively
pumps CSF caudally, creating sustained pressure elevation in the caudal
segment. In other words, the energy of the to-and-fro CSF movement is
translated via the one-way valve into the creation and sustenance of
syringomyelia.

Direction-selective resistance to CSF flow is not an imaginative assumption
but actually exists in patients. In Chiari-I malformation, the herniated
tonsils move like a ball-valve, displaced caudally during the caudal flow
and rostrally during the rostral flow. Higher velocity observed in the
phase-contrast MRI studies suggests that it selectively impedes the caudal
CSF flow more than the cranial flow. This direction-selective resistance
was demonstrated by Williams in direct measurements in Chiari-I patients
and became the basis of his theory \cite{williams1981simultaneous}.

Also, there is a possibility that some types of arachnoid pathology
function as one-way valves. In 2014, we reported a case of thoracic
arachnoid web associated with syringomyelia, in which phase-contrast MRI
detected one-way-valve-like behavior of the arachnoid web
\cite{chang2014dorsal}. We found an obliquely oriented arachnoid web that
reminded us of a one-way valve in surgery. Thus, our hypothesis may also
solve the second theoretical problem that we pointed out in the
Introduction. Namely, we may consider the presence of a one-way valve in
spinal subarachnoid space as a common mechanism underlying both
Chiari-I-related and arachnopathy-related syringomyelia.  Our theory
assumes that the arterial-pulse generated to-and-fro CSF movement provides
the energy for syrinx generation. It is compatible with Stoodley et al.'s
experimental result that syrinx maintenance is dependent on arterial
pulsation \cite{stoodley2000mechanisms}. In this study, the authors showed that

\bibliography{my_library}


\end{document}
